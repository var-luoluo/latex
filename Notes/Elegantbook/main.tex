\documentclass[lang=cn]{elegantbook}

\title{ElegantBook 示例文档}
\date{2024/03/31}
\cover{cover.jpg}
%% Begin~
\begin{document}
\maketitle


\frontmatter
\tableofcontents


\mainmatter
\chapter{Elegant\LaTeX{} 个人使用}

\section{定理环境}

\begin{definition}[一致连续]\label{def:1}
  $f(x)$ 在 区间 $I$ 上有定义. 若对于任意的 $\epsilon > 0$, 存在 $\delta = \delta(\epsilon) > 0$,%
  使得对任意的 $x_1, x_2\in I$, 只要 $|x_1 - x_2| < \delta$, 就有
  \[
    |f(x_1) - f(x_2)| < \epsilon ,
  \]
  则称函数 $f$ 在区间 $I$ 上一致连续。
  \begin{flushright}
    华师大·数学分析第五版
  \end{flushright}
\end{definition}

\begin{theorem}[一致连续性定理]\label{thm:1}
  若函数 $f$ 在闭区间 $[a,b]$ 连续, 则 $f$ 在 $[a,b]$ 一致连续.
\end{theorem}

\begin{table}[htbp]
  \centering
  \caption{定理类环境}
    \begin{tabular}{llll}
    \toprule
    环境名 & 标签名 & 前缀 & 交叉引用 \\
    \midrule
    definition & label & def   & \lstinline|\ref{def:label}| \\
    theorem & label & thm   & \lstinline|\ref{thm:label}| \\
    postulate & label & pos & \lstinline|\ref{pos:label}| \\
    axiom & label & axi & \lstinline|\ref{axi:label}|\\
    lemma & label & lem   & \lstinline|\ref{lem:label}| \\
    corollary & label & cor   & \lstinline|\ref{cor:label}| \\
    proposition & label & pro   & \lstinline|\ref{pro:label}| \\
    \bottomrule
    \end{tabular}%
  \label{tab:theorem-class}%
\end{table}%


\section{其他环境}

\begin{exercise}\label{exer:43}
设 $f \notin\in L(\mathcal{R}^1)$,$g$ 是 $\mathcal{R}^1$ 上的有界可测函数。证明函数
\begin{equation}
   \label{ex:1}
   I(t) = \int_{\mathcal{R}^1} f(x+t)g(x)dx \quad t \in \mathcal{R}^1
\end{equation}
是 $\mathcal{R}^1$ 上的连续函数。 
\end{exercise}
\begin{solution}
即 $D(x)$ 在 $[0,1]$ 上是 Lebesgue 可积的并且积分值为零。但 $D(x)$ 在 $[0,1]$ 上不是 Riemann 可积的。
\end{solution}

\begin{proof}
即 $D(x)$ 在 $[0,1]$ 上是 Lebesgue 可积的并且积分值为零。但 $D(x)$ 在 $[0,1]$ 上不是 Riemann 可积的。
\end{proof}

\begin{table}[htbp]
  \centering
  \caption{其他环境}
    \begin{tabular}{llll}
    \toprule
    环境名 & 标签名 & 前缀 & 交叉引用 \\
    \midrule
    exercise & label & exer   & \lstinline|\ref{exer:label}| \\
    solution & label & -- & -- \\
    proof & label & -- & -- \\
    note & label & -- & -- \\
    conclusion & label & -- & -- \\
    \bottomrule
    \end{tabular}%
  \label{tab:theorem-class}%
\end{table}%

\section{公式}
\begin{equation}
    \label{inter2}
    \int_0^1 D(x)dx = \int_0^1 \chi_{Q_0} (x) dx = m(Q_0) = 0
\end{equation}


    
\begin{align}
  \lim_{x\to 0} \frac{e^x-(1+2x)^\frac{1}{2}}{\ln (1+x^2)} &= \lim_{x\to 0} \frac{e^x-(1+2x)^\frac{1}{2}}{x^2}\notag\\
  &= \lim_{x\to 0} \frac{e^x-(1+2x)^{-\frac{1}{2}}}{2x}\notag\\
  &= \lim_{x\to 0} \frac{e^x+(1+2x)^{-\frac{3}{2}}}{2}\notag\\
  &= 1
\end{align}


\section{图像}
\end{document}